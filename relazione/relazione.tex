\documentclass[12pt,a4paper]{book}
\usepackage[utf8]{inputenc}
\usepackage[pdftex]{graphicx}
\usepackage[italian]{babel}
\usepackage{fixltx2e,bold-extra,geometry,
    amssymb,amsmath,mathtools, microtype,url,cite}
\usepackage[bookmarks=true, hidelinks, pdftitle={
RELAZIONE FINALE SISTEMI DIGITALI M},
pdfauthor={De Nardi-Tornatore}]{hyperref}
% \pagestyle{headings}
\graphicspath{{images/}}

\usepackage{float} %use H in pictures

\usepackage{pgfplots}

% Default fixed font does not support bold face
\DeclareFixedFont{\ttb}{T1}{txtt}{bx}{n}{9} % for bold
\DeclareFixedFont{\ttm}{T1}{txtt}{m}{n}{9} % for normal

% Custom colors
\usepackage{color}
\definecolor{deepblue}{rgb}{0.45,0.95,0.36}
\definecolor{deepred}{rgb}{0,0,0}
\definecolor{deepgreen}{rgb}{0.67,0.31,0.31}
\definecolor{deepblack}{rgb}{0.11,0.11,0.11}
\definecolor{lightblue}{rgb}{0.77, 0.89, 0.93}

\usepackage{listings}

% Python style for highlighting
\newcommand\pythonstyle{\lstset{
		backgroundcolor=\color{deepblack},
		language=Python,
		basicstyle=\ttm\color{white},
		morekeywords={self},              % Add keywords here
		keywordstyle=\ttb\color{deepblue},
		emph={MyClass,__init__},          % Custom highlighting
		emphstyle=\ttb\color{deepred},    % Custom highlighting style
		stringstyle=\color{deepgreen},
		commentstyle=\ttm\color{lightblue},
		frame=tb,                         % Any extra options here
		showstringspaces=false
}}


% Python environment
\lstnewenvironment{python}[1][]
{
	\pythonstyle
	\lstset{#1}
}
{}

% Python for external files
\newcommand\pythonexternal[2][]{{
		\pythonstyle
		\lstinputlisting[#1]{#2}}}

% Python for inline
\newcommand\pythoninline[1]{{\pythonstyle\lstinline!#1!}}


%%%%%%%%%%%%%%%%% New Commands %%%%%%%%%%%%%%%%%%%%%%%%%%%%%%%%%%%%%%%%%%%%%%%%%
%
\newcommand{\intentblankpage}{
%     Leaves a blank page
    \newpage
    \null
    \vfill
    \thispagestyle{empty}
    \begin{center}
%         \textit{This page intentionally left blank.}
        \textit{Questa pagina \`e lasciata intenzionalmente bianca.}
    \end{center}
    \newpage
}
% 
% Writes intentionally blank page when there is a \newpage on the left page of
% a book.
\makeatletter
    \def\cleardoublepage{\clearpage%
        \if@twoside
            \ifodd\c@page\else
                \vspace*{\fill}
                \hfill
                \begin{center}
%                     \textit{This page intentionally left blank.}
                    \textit{}
                \end{center}
                \thispagestyle{empty}
                \newpage
                \if@twocolumn\hbox{}\newpage\fi
            \fi
        \fi
    }
\makeatother
%
% Length to set margins for 65 chars
\newlength{\sixtyfivecharwidth}
\settowidth{\sixtyfivecharwidth}{
    \normalfont abcdefghijklmnopqrstuvwxyzabcdefghijklmnopqrstuvwxyz1234567890123}
% 
%Integral with the limit below(mathrlap)
% \newcommand{\intlimr}[1]{\ensuremath{\int \limits_{\mathrlap{#1}}}}
% 
% This redefine could cause big issues! See: 
% http://tex.stackexchange.com/questions/248421/use-mathclap-as-default-in-limits-of-integration
% \let\oldlimits\limits
% \def\limits_#1{\oldlimits_{\mathclap{#1}}}
\def\mclimits_#1{\limits_{\mathclap{#1}}}
%%%%%%%%%%%%%%%% End New Commands %%%%%%%%%%%%

\begin{document}
    \pagenumbering{Roman}
    \input{./titlepage}
    
    \tableofcontents
    
    \chapter*{Introduzione} \pagenumbering{arabic}
    \addcontentsline{toc}{chapter}{Introduzione}
    \label{CH:Intro}
    La musica ha assunto un ruolo di primo piano nella storia dell'uomo. Forse è la voglia di esprimere i propri sentimenti e le proprie emozioni che hanno obbligato l'essero umano a comporre sempre nuovi brani. Senza alcun'ombra di dubbio, la chitarra è uno degli strumenti più usati. Per esempio, quando si va in campeggio e alla sera ci si siede davanti al falò, l'aria si riempe delle sue note.\\ Il tema libero lasciato dai professori ha permesso di approfondire ma soprattutto conoscere meglio il vastissimo campo della musica.  \\ Il progetto ha l'obiettivo di trascrivere in modo automatico le tablature per chitarra in modo che anche chi non è in grado di suonare questo strumento lo possa fare. Il tutto funziona tramite l'ausilio di un semplice \textit{smartphone}. \\

La relazione è articolata come segue:
\begin{itemize}
	\item nel \textbf{capitolo 1} viene descritta brevemente come è fatta la chitarra;
	\item nel \textbf{capitolo 2} viene introdotto il dominio in cui lavoreremo;
	\item nel \textbf{capitolo 3} viene descritto il modello della nostra rete, l'addestramento che è stato eseguito e la sua accuratezza;
	\item nel \textbf{capitolo 4} viene realizzata l'implementazione sul dispositivo \textit{embedded}; nel nostro caso sono gli \textit{smartphone};
	\item il \textbf{capitolo 5} conclude la relazione presentando i risultati ottenuti e gli obiettivi raggiunti.
\end{itemize}
La relazione è stata scritta come un \textit{diario} mettendo in risalto tutti i passaggi, i tentativi e i problemi che si sono verificati.
    
    \chapter{Chitarra}
    \label{CH:Teoria}
    La chitarra ha una lunga tradizione che affonda le sue radici addirittura al tempo degli arabi. I primi esemplari risalgono al tredicesimo secolo. Inizialmente dotata di quattro corde, si è accresciuta nel Rinascimento di un'altra corda, arrivando poi nel periodo Barocco all'attuale numero di sei.\\
\newline
La chiatarra è composta da due parti principali:
\begin{itemize}
	\item il manico, su cui si trova la tastiera e che termina con la paletta la quale ospita le meccaniche per l'accordatura;
	\item la cassa di risonanza o tavola armonica con una cavità centrale, che serve ad amplificare il suono prodotto dalle corde.
\end{itemize}

\begin{figure}[H]
	\centering
	\includegraphics[scale=0.50]{./images/img14.jpg}
\end{figure}

\subsection{Corde}
Le corde delle chitarre moderne sono sei e sono ordinate dall'alto verso il basso nel seguente modo:
\begin{itemize}
	\item la prima corda corrisponde alla nota Mi cantino (e);
	\item la seconda corrisponde alla nota Si (B);
	\item la terza corrisponde alla nota Sol (G);
	\item la quarta corrisponde alla nota Re (D);
	\item la quinta corrisponde alla nota La (A);
	\item la sesta corrisponde alla nota Mi basso (E).
\end{itemize}
L'ultima corda dell'elenco è quella più spessa, mentre la prima è la più sottile.
\subsection{Tasti}
Sul manico della chitarra c’è la tastiera. Si chiama tastiera proprio perchè ci sono i tasti. Quest'ultimi sono delimitati da delle barrette di metallo e ognuno di essi corrisponde a una nota. Dunque, se abbiamo una chitarra a 19 tasti, possiamo fare 19 note diverse per ogni corda.\\
La distanza tra due tasti della stessa corda prende il nome di \textbf{semitono}. Ad esempio, se premiamo la sesta corda in corrispondenza del La, poi premendo la corda al tasto adiacente più vicino alla cassa di risonanza (un semitono più alto) ascolteremo un La\#.\\
Se non premiamo nessun tasto la corda si dice che è suonata a vuoto. Le sei corde suonate a vuoto devono emettere dei suoni ben precisi: la chitarra deve essere quindi accordata. L'accordatura classica delle sei corde, ovvero la nota che devono suonare le corde a vuoto (dal basso all'alto), è la seguente: Mi, La, Re, Sol, Si, Mi.\\
Conoscendo il suono prodotto dalle sei corde suonate a vuoto e sapendo che ogni nota suonata ad un tasto dista di un semitono dalla nota suonata al tasto adiacente possiamo mappare tutta la tastiera della chitarra.
\begin{figure}[H]
	\centering
	\includegraphics[scale=0.60]{./images/img13.jpg}
\end{figure}

\subsection{Tab}
La \textit{tab} è una rappresentazione delle corde della chitarra. Una tablatura è solitamente scritta usando sei linee orizzontali, ognuna corrispondente a una corda.\\
Al contrario dei normali spartiti, su una tablatura non ci sono le note da suonare ma si trovano le indicazioni su dove mettere le dita. I numeri sulle linee corrispondono ai tasti della tastiera. Ad esempio, un "1" sulla prima corda, indica di suonare il Mi cantino tenendo premuto il primo tasto.
Se il numero è superiore a zero, bisogna premere il tasto corrispondente quando si suonerà quella corda. Se troviamo uno \textbf{zero} allora si suona la corda a vuoto, senza premere alcun tasto.
\begin{figure}[H]
	\centering
	\includegraphics[scale=0.50]{./images/img15.png}
\end{figure}

Spesso leggendo una tablatura si trovano dei numeri che sono allineati verticalmente. In questo caso si premono più tasti contemporaneamente. Le tab vanno lette come libri cioè da sinistra a destra.
\begin{figure}[H]
	\centering
	\includegraphics[scale=0.50]{./images/img16.png}
\end{figure}
    
    \chapter{Trasformata a Q Costante}
    \label{CH:Teoria}
    Le note musicali si possono classificare nel seguente modo:
\begin{figure}[H]
	\centering
	\includegraphics[scale=0.15]{./images/img4.jpg}
\end{figure}
La lettera a sinistra identifica la nota musicale mentre il numero a destra rappresenta la sua frequenza.

In musica, un'\textbf{ottava} è l'intervallo di otto note posizionate a frequenza diversa nella scala musicale. Le frequenze intermedie sono altre sei note. Per esempio, il \textit{La3} (A4) ha frequenza di 440 Hz, il \textit{La} posto un'ottava sopra ha frequenza 880 Hz, quello un'ottava sotto ha frequenza 220 Hz.

Se si rappresentassero le prime sei ottave della nota C potremmo vedere che la sua frequenza raddoppia ad ogni ottava. \\
\newline
I tasti che intercorrono fra gli estremi della stessa ottava (esempio \textit{Do3}-\textit{Do4}) sono dodici note per cui la frequenza deve raddoppiare ogni dodici note. Si può rappresentare quanto detto dalla seguente formula: \\

\begin{center}
	\begin{math}
		F_{k}=440Hz \cdot 2^{k \over 12}
	\end{math}
\end{center}
Nel campo della musica viene utilizzata la trasformata a Q costante, a discapito della più nota trasformata di Fourier, proprio per la sua natura esponenziale. Inoltre, l'accuratezza della trasformata a Q costante è analoga alla scala logaritmica e imita l'orecchio umano, avendo una risoluzione di frequenza più alta a quelle più basse e una risoluzione più bassa alle frequenze più alte. Infatti, dal seguente grafico si può notare la natura esponenziale della funzione:
\begin{center}
	\begin{tikzpicture}
	\begin{axis}[ 
		xlabel=$k$,
		ylabel={$frequenza$}
		] 
		\addplot {440*2^x/12}; 
	\end{axis}
\end{tikzpicture}
\end{center}
    
    \chapter{Architettura e addestramento del modello}
    \label{CH:Val_num}
    \section{Set dati GuitarSet}
\textbf{tempo di lavoro:} \\
\newline
Fortunatamente, su Internet abbiamo trovato un \textit{data set} di file audio di chitarra già pronto su cui lavorare. Il \textit{GuitarSet}, chiamato così dal suo creatore, è costituito dai file audio e dai suoi \textbf{tab}.\\
Questo \textit{data set} contiene 360 estratti di canzoni della durata di circa 30 secondi l'uno. Essi sono il risultato delle seguenti combinazioni:
\begin{itemize}
	\item 6 persone suonano ciascuno gli stessi 30 fogli
	\item Vengono registrate 2 versioni diverse: comping e soloing
\end{itemize}
I 30 fogli sono generati da una combinazione di
\begin{itemize}
	\item \textbf{5 stili}: Rock, Cantautore, Bossa Nova, Jazz e Funk
	\item \textbf{3 progressioni}: 12 Bar Blues, Autumn Leaves e Pachelbel Canon.
	\item \textbf{2 Tempi}: lento e veloce.
\end{itemize}
Gli estratti sono registrati sia con il pickup esafonico che con un microfono a condensatore Neumann U-87.
Ci sono tre registrazioni audio per ogni estratto:
\begin{itemize}
	\item \textbf{hex}: file wav originale a 6 canali dal pickup esafonico;
	\item \textbf{hex\_cln}: file wav esadecimali con rimozione delle interferenze applicata;
	\item \textbf{mic}: registrazione monofonica dal microfono di riferimento
\end{itemize}
Noi abbiamo usato registrazioni di tipo \textbf{mic} perchè sono quelle che più si avvicinano al caso delle registrazioni tramite microfono dello smartphone.\\
\newline
Ciascuno dei 360 estratti ha anche un file .jams che memorizza 16 annotazioni da cui prenderemo le tab:
\begin{itemize}
	\item Intonazione:
	\begin{itemize}
		\item 6 annotazioni \textit{pitch\_contour} (1 per stringa);
		\item 6 annotazioni \textit{midi\_note} (1 per stringa);
	\end{itemize}
	\item Beat e tempo:
	\begin{itemize}
		\item 1 annotazione \textit{beat\_position};
		\item 1 annotazione del tempo;
	\end{itemize}
	\item Accordi:
	\begin{itemize}
		\item 2 annotazioni di accordi (istruite ed eseguite).
	\end{itemize}
\end{itemize}
Noi useremo le annotazioni \textit{midi\_note} che contengono le tab.
\subsection{Ricavare le tab dai file .jams}
Innanzitutto calcoliamo i \textit{frame} per ogni file audio, così da poter ricavare un immagine e la corrispondente tab per ogni frame.
Per calcolare l'istante di tempo per ogni frame utilizziamo la funzione \textit{get_times()}
\pythonexternal{./codes/times.py}

Successivamente calcoliamo 
\pythonexternal{./codes/labels.py}


\textit{NumPy} è una libreria che aggiunge supporto a grandi matrici e array multidimensionali insieme a una vasta collezione di funzioni matematiche di alto livello per poter operare efficientemente su queste strutture dati.\\
\newline
I dati sono stati compressi in \textit{numpy array} (.npz) per organizzarli meglio altrimenti avremmo avuto un'immagine e una label per ogni file audio.

\pythonexternal{./codes/labels.py}

\subsection{Trasformata a Q Costante in Librosa}
\textit{Librosa} è una libreria per la musica e l'analisi audio. Fornisce gli elementi costitutivi necessari per creare sistemi di recupero delle informazioni musicali.\\
\newline
La trasformazione a Q Costante può essere facilmente applicata ai file audio utilizzando la libreria \textit{librosa}.

\pythonexternal{./codes/librosa.py}
Ogni file audio è stato diviso ogni tot secondi in modo da stampare le note che sono state suonate sulla chitarra sempre tramite la libreria \textit{librosa}. L'obiettivo è quello di prendere l'immagine appena ottenuta e di darla in input alla rete CNN. Sperimentalmente, abbiamo visto che per ottenere un buon valore di accuratezza nella rete, le immagini devono essere stampate ogni 0.2 secondi.

\begin{figure}[H]
	\centering
	\includegraphics[scale=0.90]{./images/img7.png}
\end{figure}
Se il tempo fosse stato inferiore avremmo avuto immagini di file audio con solo note singole e quando la rete avrebbe dovuto riconoscere più note non sarebbe stata in grado di farlo. Nell'immagine seguente è possibile vedere che le lettere sulle note si ripetono. Ad esempio, la lettera F si trova sulla corda F e D. Se la mano è posizionata sulla corda A, è impossibile che si riesca a suonare la F della nota G. Dunque, un istante di tempo troppo corto non aiutava la rete a riconoscere la nota giusta.

\begin{figure}[H]
	\centering
	\includegraphics[scale=0.20]{./images/img12.jpg}
\end{figure}

L'immagine risultante da \textit{librosa} ha dimensione 192x9.\\

%da spostare
I tab sono matrici 6x19 e hanno caratteristica di essere \textit{one-shot} cioè si può trovare un uno solo in ogni riga. Ad ogni uno presente nella riga corrisponde a un nota. Le sei righe della matrice corrispondono alle sei note della chitarra mentre i tasti sono diciotto. La colonna in più, più precisamente la prima, indica che nella prima colonna la corda non è stata toccata.

\subsection{Pre-elaborare i dati}

% keras e
	% https://www.pyimagesearch.com/2019/10/21/keras-vs-tf-keras-whats-the-difference-in-tensorflow-2-0/

\section{Modello della rete}
\textbf{tempo di lavoro:} \\
\newline
\subsection{Uso di Keras}
Keras consente di implementazione algoritmi basati su reti neurali. Permette di sviluppare e prototipare in maniera semplice e veloce modelli nell’ambito del machine learning e del deep learning.\\
\pythonexternal{./codes/norm2.py}

\pythonexternal{./codes/modello.py}
\subsection{Compilazione del modello}
\textbf{tempo di lavoro:} \\
\newline
\pythonexternal{./codes/compilare.py}
\subsection{Addestramento del modello}
\textbf{tempo di lavoro:} \\
\newline
Dopo diverse prove sperimentali, abbiamo deciso di eseguire il modello per 30 epoche:

Questo modello raggiunge una precisione di circa 0,91 (o 91\%) sui dati di addestramento.
\begin{figure}[H]
	\centering
	\includegraphics[scale=0.70]{./images/img5.png}
\end{figure}

\section{Accuratezza del modello}
\textbf{tempo di lavoro:} \\
\newline
Abbiamo confrontato le prestazioni del modello sul set di dati di test:

\pythonexternal{./codes/accuratezza.py}
\begin{figure}[H]
	\centering
	\includegraphics[scale=0.70]{./images/img6.png}
\end{figure}
    
    \chapter{Implementazione su dispositivo Embedded}
    \label{CH:Val_num}
    La scelta è caduta sui dispositivi mobili. Questa decisione è stata vincolante perchè disponevamo di solo queste risorse \textit{hardware}.

\section{Sviluppo applicazione iOS}
\subsection{Conversione del modello da Keras a CoreML}
\textbf{tempo di lavoro:} 12 giorni\\
\newline
Per poter usare il modello pre-addestrato sul cellulare abbiamo dovuto convertirlo nel formato \textit{.mlmodel} in modo da poter usare il \textit{framework Core ML}.\\
\newline
\textbf{Problemi riscontrati:} durante la conversione del modello sono apparsi diversi errori che impedivano la conversione. Gli errori sono simili a quello riportato di seguito:
\begin{figure}[H]
	\centering
	\includegraphics[scale=0.60]{./images/img1.png}
\end{figure}
\textbf{Soluzioni provate:}
\begin{itemize}
	\item Abbiamo preso spunto dal codice che si trova sul blocco di lucidi visti a lezione. Esso usa il package \textit{tfcoreml}. Il focus dello \textit{script} è il seguente:
	
	\pythonexternal{./codes/coreml1.py}
	
	A questo punto serviva ottenere il file in formato \textit{.pb} da usare come \textit{input}. Questo file prende il nome di modello congelato. Prima di ottenerlo serve salvarci il modello che si ottiene con \textit{Tensorflow}. Esso è formato da quattro file:
	\begin{itemize}
		\item \textbf{model-ckpt.meta}: contiene il grafico completo (flusso di dati, le annotazioni per le variabili, le \textit{pipeline} di \textit{input} e altre informazioni);
     	\item \textbf{model-ckpt.data-0000-of-00001}: contiene tutti i valori delle variabili (pesi, segnaposto, gradienti, iperparametri, ecc.);
     	\item \textbf{model-ckpt.index}: ci sono tutti i metadati. È una tabella immutabile in cui ogni chiave è un nome di un tensore e il suo valore descrive i metadati di un tensore;
     	\item \textbf{checkpoint}: tutte le informazioni sul \textit{checkpoint}.
	\end{itemize}
      
	\pythonexternal{./codes/coreml2.py}
	
	\textit{estimator\_model.train} serve a verificare che il modello esportato in precedenza sia effettivamente funzionante.\\
	\newline
	Il modello congelato ci consente di eliminare tutte le informazioni in più che vengono salvate perchè si potrebbe ricaricare quello appena salvato e l'addestramento continua da dove era stato interrotto.
	
	\pythonexternal{./codes/freezer.py}
	
	\item Uno dei nuovi tentativi si è basato sul cambio di codice per salvare il modello: abbiamo usato il seguente codice:
	
	\pythonexternal{./codes/coreml3.py}
	
	Tuttavia, i risultati non sono stati quelli sperati.
	
	\item Cercando nella documentazione di \textit{coreml}, abbiamo scoperto che non sono previsti più aggiornamenti e consigliavano di usare un nuovo \textit{package}. Anche se fossimo stati in grado di convertirlo non avremmo potuto utilizzato su sistemi operativi \textit{iOS} maggiori di 12. La nuova libreria che abbiamo usato si chiama \textit{coremltools}.
	
	\pythonexternal{./codes/coremltools.py}
	
	\item Per non avere altri errori che ci apparivano, abbiamo usato i livelli di convoluzione presi direttamente da \textit{keras} e non da \textit{tensorflow}. E' stato molto difficile superare tutti questi ostacoli perché i messaggi di errore non aiutavano a capire bene che cosa bisognasse modificare.
	
	\item Successivamente, per aumentare l'accuratezza, è stata inserita una funziona di attivazione personalizzata. Purtroppo, anche con \textit{coremltools} non siamo stati in grado di convertirlo perchè ci appariva un errore in corrispondenza del livello della nuova funzione:
	
	\begin{figure}[H]
		\centering
		\includegraphics[scale=0.60]{./images/img11.png}
	\end{figure}
	
\end{itemize}
\textbf{Soluzione finale:} abbiamo deciso di usare \textit{Tensorflow Lite} perchè siamo riusciti a convertire il modello subito senza nessun problema.

\subsection{Conversione del modello da Keras a Tensorflow Lite}
\textbf{tempo di lavoro:} 1h\\
\newline
Come accennato nel paragrafo precedente, abbiamo convertito il modello usando \textit{Tensorflow Lite}.
\pythonexternal{./codes/tensorflowLite.py}
Il seguente codice consente di caricare il modello addestrato tramite \textit{Tensorflow} e di convertirlo nel formato .\textit{tflite} pronto per essere usato su un dispositivo mobile nel nostro caso.

\subsection{Sviluppo applicazione con Swift 5 e Xcode 12}
\textbf{tempo di lavoro:} 20 giorni\\
\newline
\textit{Swift} è un linguaggio di programmazione \textit{object-oriented} concepito per programmare sui sistemi operativi \textit{Apple}.\\
\newline
\textit{Xcode} è un ambiente di sviluppo integrato completamente sviluppato e mantenuto da \textit{Apple}, che consente di sviluppare \textit{software} per i sistemi \textit{macOS}, \textit{iOS}, \textit{watchOS} e \textit{tvOS}.\\
\newline
Abbiamo dovuto prendere un pò di familiarità con il nuovo linguaggio e il nuovo \textit{IDE} dato che non avevamo mai programmato nel mondo \textit{Apple}. La documentazione messa a disposizione agli sviluppatori è molto vasta e le solide basi apprese a ingegneria hanno fatto il resto.
La scelta è ricaduta direttamente sia all'ultima versione del linguaggio che dell'ambiente di sviluppo perché non avevamo vincoli sulla realizzazione dell'applicazione.\\
\newline
Le interfacce si realizzano in modo molto semplice perchè \textit{Xcode} consente di spostare gli elementi grafici con il \textit{mouse} e di posizionarli come si vuole. Tuttavia, è stata la parte che ha richiesto più tempo perchè li abbiamo dovuti configurare nel modo più adatto alle nostre esigenze. 
\begin{figure}[H]
	\centering
	\includegraphics[scale=0.20]{./images/img2.png}
\end{figure}
E' stata prestata anche molta attenzione a rendere compatibile l'applicazione su modelli diversi. La dimensione dello schermo influisce molto sul \textit{layout} dell'applicazione. Senza le giuste modifiche è possibile che un elemento venga nascosto o spostato.
\begin{figure}[H]
	\centering
	\includegraphics[scale=0.20]{./images/img3.png}
\end{figure}
Le due immagini precedenti mostrano chiaramente quello appena descritto: i due dispositivi sono diversi e le proporzioni vengono rispettate in entrambi.\\
\newline
Inoltre, una particolare attenzione è stata dedicata anche sulla nuova modalità che sta riscontrando un grandissimo successo: la \textit{dark mode}. Dunque, sono stati presi tutti gli accorgimenti necessari per avere sia l'app compatibile con la versione chiara che con quella scura. L'immagine successiva mostra l'app in versione \textit{dark}:
\begin{figure}[H]
	\centering
	\includegraphics[scale=0.20]{./images/img10.png}
\end{figure}

\subsection{Uso di TensorFlow Lite su iOS}
Abbiamo eseguito i seguenti passaggi:
\begin{itemize}
	\item Registrato l'\textit{app} sul sito \textit{Firebase} perchè bisogna monitorarla (passaggio obbligatorio come scritto nella guida);
	\item Scaricato e aggiunto il file di configurazione che si ottiene dopo la registrazione su \textit{Firebase};
	\item Aggiunto \textit{Firebase} all'\textit{app};
	\item Inizializzato \textit{Firebase} nel progetto \textit{iOS} e usate le sue \textit{API} per usare il modello sullo \textit{smartphone}.
\end{itemize}

\subsection{Uso di un server per l'uso della libreria Librosa}
\textbf{tempo di lavoro:} 5 giorni\\
\newline
\textbf{Problemi riscontrati:} putroppo non sono state trovate librerie in grado di convertire il file audio nella trasformata a Q costante.\\
\newline
%
\textbf{Soluzioni provate:}
\begin{itemize}
	\item Abbiamo provato ad usare la libreria \textit{PythonKit} senza successo perchè sui dispositivi \textit{iOS} manca l'interprete \textit{Python}.\\
\end{itemize}
%
\textbf{Soluzione finale:} per questo motivo ci siamo serviti di un \textit{server} che prende in ingresso la registrazione che è stata effettuata dallo \textit{smartphone} e restituisce in uscita le immagini del file audio. Ovviamente la soluzione non è efficiente ma ai fini del progetto può andare più che bene. La predizione viene eseguita sul dispositivo e \textbf{non} sul \textit{server}.\\
\newline
Il \textit{server} è stato scritto grazie al \textit{framework} di \textit{Python} che si chiama \textit{flask}

\pythonexternal{./codes/flask.py}

\section{Sviluppo applicazione Android}
\subsection{Conversione del modello da Keras a Tensorflow Lite}
\textbf{tempo di lavoro:} 1h\\
\newline
Tutto ha funzionato al primo colpo senza nessun problema.\\
\pythonexternal{./codes/tensorflowLite.py}
Il seguente codice consente di caricare il modello addestrato tramite \textit{Tensorflow} e di convertirlo nel formato .\textit{tflite} pronto per essere usato su un dispositivo mobile nel nostro caso.

\subsection{Sviluppo applicazione con Java 1.8 e Android Studio 4.1.2}
\textbf{tempo di lavoro:} 12 giorni \\
\newline
\textit{Java} è una piattaforma che consente di eseguire i programmi scritti in questo linguaggio.\\
\newline
\textit{Android Studio} è un ambiente di sviluppo integrato per lo sviluppo per la piattaforma \textit{Android}.\\
\newline
L'applicazione che è stata realizzata da un punto di vista estetico è uguale a quella su \textit{iOS}. Cambiano solo leggeri particolari che differenziano i due mondi.\\
\newline
\textbf{Problemi riscontrati:} putroppo non sono state trovate librerie in grado di convertire il file audio nella trasformata a Q costante.\\
\newline
%
\textbf{Soluzioni provate:}
\begin{itemize}
	\item Abbiamo usato la soluzione del \textit{server} come in \textit{iOS}.\\
\end{itemize}
%
\textbf{Soluzione finale:} grazie ai ricevimenti fatti con il professore, ci è stato consigliato di usare \textit{chaquopy}.
    
    \chapter{Conclusioni}
    \label{CH:Concl}
    Sulle due applicazioni sono stati eseguiti diversi \textit{test} e possiamo affermare che funziona tutto correttamente.
Siamo molto soddisfatti del progetto che è stato realizzato. Siamo consapevoli che abbiamo esplorato solo una piccola parte di questa vastissima materia ma quello che abbiamo appreso sarà sicuramente usato come base di partenza in futuri progetti.\\
    
%     \nocite{*}
    \bibliographystyle{IEEEtran}
    %%%%%%%%%%% Example 1%%%%%%%%%%%%%%%%%%%%%%%%%%
    \begin{thebibliography}{100}  % 100 is a random guess of the total number of %references
    
    \bibitem{HK} \textit{Chitarra classica}. (s.d.). Wikipedia, l'enciclopedia libera. Ultimo accesso: 28 febbraio 2021, https://it.wikipedia.org/wiki/Chitarra
    
    \bibitem{HK} \textit{I tasti della chitarra}. (s.d.). TestoeAccordi. Ultimo accesso: 26 febbraio 2021, https://www.testoeaccordi.it/menu/tasti.htm
    
    \bibitem{HK} \textit{Note manico chitarra} [Image]. (s.d.). Videocorsochitarra. Ultimo accesso: 27 febbraio 2021, https://videocorsochitarra.it/note-manico-chitarra/
    
    \bibitem{HK} Savage. N. (s.d.). \textit{Come Leggere Tablature per Chitarra}. Wikihow. https://www.wikihow.it/Leggere-Tablature-per-Chitarra
    	
    \bibitem{Boney96} Q. Xi, R. Bittner, J. Pauwels, X. Ye, and J. P. Bello, "Guitarset: A Dataset for Guitar Transcription", \emph{in 19th International Society for Music Information Retrieval Conference}, Paris, France, Sept. 2018.
    
    \bibitem{MG} Note Frequency Chart (Download) [Image], Soundonsound.com, Dec 03, 2018 11:32 am.
    
    \bibitem{HK} Drexel University, ExCITe Center, \emph{Expressive and Creative Interaction Technologies}, NEMISIG 2019
    
    \bibitem{HK} \textit{Utilizza un modello TensorFlow Lite per inferenza con ML Kit su iOS}. (s.d.). Firebase. Ultimo accesso: 03 marzo 2021, https://firebase.google.com/docs/ml-kit/ios/use-custom-models

	\end{thebibliography}
	%%%%%%%%%%%%% end %%%%%%%%%%%%%%%%%%%%%%%%%%%%%%%
\end{document}
